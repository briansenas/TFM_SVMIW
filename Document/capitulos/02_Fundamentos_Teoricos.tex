\chapter{Fundamentos Teóricos}
\section{Modelado de cámara}
La formación de imágenes en sistemas de visión por computadora se basa en la proyección perspectiva, 
un modelo geométrico que describe cómo los puntos del espacio tridimensional (3D), habitualmente denominado
las coordenadas del mundo, se proyectan sobre un plano bidimensional (2D), el plano de la imagen. 
En la Figura~\ref{fig:PinHole} se observa el modelo fundamental, de cámara estenopeica, 
que idealiza la cámara como un sistema sin lentes donde los rayos de luz pasan por un solo punto. 
Un punto perteneciente al espacio 3D, $P=(X,Y,Z)$, se proyecta
sobre el plano de imagen utilizando las relaciones definidas en la Ecuación~\ref{eq:2DProyection},
tras aplicar la semejanza entre triángulos~\cite{Hartley2004,VisionBookMIT}. 
\begin{equation}\label{eq:2DProyection}
    x=f\cdot\frac{X}{Z}, \; y=f\cdot\frac{Y}{Z}
\end{equation}

\begin{figure}[!ht]
\begin{center}
    \includegraphics[width=0.7\textwidth]{imagenes/chapter2/pinhole-model}
\end{center}
\caption{
    Se observa la geometría subyaciente del modelo de cámara estenopeico sacada de~\cite{Hartley2004}.
    La distancia entre el punto principal, P, y el centro de la cámara se denomina distancia focal, f, y
    nos permite obtener la relación entre la coordenadas del plano de imagen y del punto en las coordenadas 
    del mundo utilizando trigonometría.
}
\label{fig:PinHole}
\end{figure}

La Ecuación~\ref{eq:2DProyection} define la proyección de perspectiva, donde objetos distantes aparentan más pequeños,
con un escalado inverso a su distancia en el eje Z. Este modelo también se aplica a la visión humana.
Esta transformación se puede expresar de forma matricial mediante coordenadas homogéneas. Si $P=(X,Y,Z,1)^T$ y $p=(x, y, 1)^T$, 
la proyección se puede expresar acorde a la Ecuación~\ref{eq:ProyectionFormula}.

\begin{equation}\label{eq:ProyectionFormula}
    s\begin{bmatrix}x\\y\\1\end{bmatrix} = K\cdot\left[R|t\right]\cdot\begin{bmatrix}X\\Y\\Z\\1\end{bmatrix}
\end{equation}

Donde $K$ es la matriz intrínseca de la cámara, que contiene los parámetros internos como la distancia focal y el 
punto principal, véase la Ecuación~\ref{eq:KMatrix}. $[R|t]$ es la concatenación de la matriz de rotación R y
el vector de traslación t, que definen la posición y orientación de la cámara respecto al eje del mundo (parámetros extrínsecos).
Por último, $s$ sería un factor de escala. Este modelo permite modelar coordenadas 3D del mundo a píxeles de la imagen, véase la 
Figura~\ref{fig:WorldToImageCoordinates}. 

\begin{figure}[!ht]
\begin{center}
    \includegraphics[width=0.5\textwidth]{imagenes/chapter2/world_and_camera_coordinates}
\end{center}
\caption{Visualización de las transformaciones necesarias, $[R|t]$, para transformar las coordenadas 3D a coordenadas 2D, sacada de~\cite{VisionBookMIT}.}
\label{fig:WorldToImageCoordinates}
\end{figure}

\begin{equation}\label{eq:KMatrix}
    K = \begin{bmatrix}
        f & q_x & cx &\\
        q_y & f & cy &\\
        0 & 0 & 1 &\\
    \end{bmatrix}
\end{equation}
Donde $f$ es la distancia focal, $(q_x, q_y)$ son distorsiones de inclinación y $(c_x, c_y)$ es la correspondencia al punto principal (central) del sensor de la imagen.
Los parámetros $f$ y $c$ se pueden estimar a partir las características de las cámaras, si se conoce dicha información, o mediante 
métodos de calibración. Con la Figura~\ref{fig:ImageSensor} se observa que la posición de los píxeles escala 
con la distancia focal, $f$, de forma que se utiliza una constante $a$ en su lugar con la relación física con 
la cámara dada por la Ecuación~\ref{eq:EstimateAlpha}.
\begin{equation}\label{eq:EstimateAlpha}
    a = f\frac{N}{w}
\end{equation}
Donde $N$ es la longitud de la imagen en píxeles y $w$ es la longitud del sensor de la cámara.
No obstante, en caso de no poseer información al respecto del sensor se puede utilizar patrones de imágenes conocidas,
como un tablero de ajedrez, para estimar-los.
Finalmente, hasta el momento no se ha introducido ningún parámetro de distorsión al modelo de cámara. Sin embargo, 
existen diferentes distorsiones que pueden ocurrir como inclinación o distorsiones radiales. Estos son parámetros
adicionales que deben estimarse para obtener una re-proyección precisa.
\begin{figure}
\begin{center}
    \includegraphics[width=0.5\textwidth]{imagenes/chapter2/pinhole_and_sensor}
\end{center}
\caption{Proyección de una imagen sobre el sensor de una cámara~\cite{VisionBookMIT}.}
\label{fig:ImageSensor}
\end{figure}
\section{Metrología de vista única}
\section{Aprendizaje automático y profundo}
\subsection{Aprendizaje automático}
\subsection{Aprendizaje profundo}
\section{Detección de objetos}
