\chapter{Introducción}
\section{Definición del Problema}   
\section{Motivación}
\section{Objetivos}
El objetivo principal de este Trabajo Fin de Máster (TFM) es desarrollar 
un \textbf{sistema de estimación automática de la estatura de individuos a partir de una sola imagen}, 
utilizando técnicas de metrología monocular. Los objetivos específicos incluyen:
\begin{enumerate}
    \item Revisar el estado del arte en técnicas de estimación de medidas antropométricas a partir de imágenes.
    \item Implementar y validar un algoritmo basado en principios de metrología visual que estime con precisión la estatura de una persona.
    \item Probar el sistema en diferentes contextos y con variabilidad en las condiciones de iluminación, perspectiva y posición de las personas.
\end{enumerate}
\section{Planificación del proyecto}
Al planificar el proyecto, es fundamental tener en cuenta que el TFM 
tiene una carga de 12 créditos ECTS, donde cada 
crédito representa aproximadamente 25 horas de trabajo. 
En total, se estima que se necesitarán alrededor de 300 horas para llevar a cabo 
el proyecto.  Considerando que el segundo cuatrimestre tiene aproximadamente 20 semanas, 
se requerirá dedicar al TFM unas 15 horas por semana, lo cual equivaldría a unas 3 horas 
diarias durante 5 días a la semana.

Dada la naturaleza del proyecto, cuya complejidad en términos de requisitos y tamaño 
del equipo de trabajo no es elevada, se utilizará un ciclo de vida en cascada~\cite{ModeloEnCascada} con realimentación.
Dicho ciclo de vida se basa en un desarrollo lineal divido en fases, en el cuál antes 
de empezar la siguiente fase se evalúa los resultados obtenidos en la fase anterior. 
De esta forma, se pueden adaptar las fases siguientes a los posibles 
nuevos hallazgos de las fases anteriores.
Las fases del ciclo de vida son: 
\begin{itemize}
  \item Análisis de requisitos: Consiste en reuniones iniciales con los clientes, 
    en este caso los directores del TFM. Se organiza el análisis bibliográfico 
    del problema de metrología monocular, se plantean los objetivos y criterios de aceptación.
  \item Diseño: Consiste en la investigación y selección de métodos conforme 
    al análisis anterior, tanto para la resolución como la validación de la solución. 
    Así como pruebas preliminares y diseño del software de experimentación. 
  \item Implementación: Consiste en la adaptación de las técnicas encontradas, 
      búsqueda de puntos de mejoría y desarrollo del producto mínimo viable. 
  \item Pruebas: Realización de diversos experimentos de validación, 
      comparando desde la metrología con parámetros estimados con una técnica de calibración manual estándar, 
      calibración basada en características de las cámaras y las estimadas por el algoritmo final.
\end{itemize}
\begin{table}[htp]
\centering
\resizebox{\textwidth}{!}{%
\begin{tabular}{|c|c|llll|llll|lllll|llll|llll|}
\hline
\rowcolor[HTML]{FFC702} 
\cellcolor[HTML]{FFC702} & \cellcolor[HTML]{FFC702} & \multicolumn{4}{c|}{\cellcolor[HTML]{FFC702}\textbf{Mayo}} & \multicolumn{4}{c|}{\cellcolor[HTML]{FFC702}\textbf{Junio}} & \multicolumn{5}{c|}{\cellcolor[HTML]{FFC702}\textbf{Julio}} & \multicolumn{4}{c|}{\cellcolor[HTML]{FFC702}\textbf{Agosto}} & \multicolumn{4}{c|}{\cellcolor[HTML]{FFC702}\textbf{Septiembre}} \\ \cline{3-23} 
\rowcolor[HTML]{FFC702} 
\multirow{-2}{*}{\cellcolor[HTML]{FFC702}\textbf{Tarea}} & \multirow{-2}{*}{\cellcolor[HTML]{FFC702}\begin{tabular}[c]{@{}c@{}}\textbf{Semanas -}\\ \textbf{Horas}\end{tabular}} & \multicolumn{1}{c}{\cellcolor[HTML]{FFC702}07} & \multicolumn{1}{c}{\cellcolor[HTML]{FFC702}14} & \multicolumn{1}{c}{\cellcolor[HTML]{FFC702}21} & \multicolumn{1}{c|}{\cellcolor[HTML]{FFC702}28} & \multicolumn{1}{c}{\cellcolor[HTML]{FFC702}04} & \multicolumn{1}{c}{\cellcolor[HTML]{FFC702}11} & \multicolumn{1}{c}{\cellcolor[HTML]{FFC702}18} & \multicolumn{1}{c|}{\cellcolor[HTML]{FFC702}25} & \multicolumn{1}{c}{\cellcolor[HTML]{FFC702}02} & \multicolumn{1}{c}{\cellcolor[HTML]{FFC702}09} & \multicolumn{1}{c}{\cellcolor[HTML]{FFC702}16} & \multicolumn{1}{c}{\cellcolor[HTML]{FFC702}23} & \multicolumn{1}{c|}{\cellcolor[HTML]{FFC702}30} & \multicolumn{1}{c}{\cellcolor[HTML]{FFC702}06} & \multicolumn{1}{c}{\cellcolor[HTML]{FFC702}13} & \multicolumn{1}{c}{\cellcolor[HTML]{FFC702}20} & \multicolumn{1}{c|}{\cellcolor[HTML]{FFC702}27} & \multicolumn{1}{c}{\cellcolor[HTML]{FFC702}04} & \multicolumn{1}{c}{\cellcolor[HTML]{FFC702}11} & \multicolumn{1}{c}{\cellcolor[HTML]{FFC702}18} & \multicolumn{1}{c|}{\cellcolor[HTML]{FFC702}25} \\ \hline
Análisis de Requisitos & 4 - 60 & \cellcolor[HTML]{9B9B9B} & \cellcolor[HTML]{9B9B9B} & \cellcolor[HTML]{9B9B9B} & \cellcolor[HTML]{9B9B9B} &  &  &  &   &  &  &  &  &  &  &  &  &  &  &  & &  \\ \cline{1-1}
Diseño & 4 - 60  &  &  &  & & \cellcolor[HTML]{9B9B9B} & \cellcolor[HTML]{9B9B9B} & \cellcolor[HTML]{9B9B9B} & \cellcolor[HTML]{9B9B9B}  & \cellcolor[HTML]{9B9B9B} &  &  & &  &   &  &  & &  &  &  &  \\ \cline{1-1}
Implementación & 6 - 90 &  &  &  &  &  &  &  &  & & \cellcolor[HTML]{9B9B9B} & \cellcolor[HTML]{9B9B9B} & \cellcolor[HTML]{9B9B9B} & \cellcolor[HTML]{9B9B9B} & \cellcolor[HTML]{9B9B9B}  & &  &  &  &  &  &  \\ \cline{1-1}
Pruebas & 6 - 90 &  &  &  &  &  &  &  &  &  &  &  &   &  & & \cellcolor[HTML]{9B9B9B} & \cellcolor[HTML]{9B9B9B} & \cellcolor[HTML]{9B9B9B} & \cellcolor[HTML]{9B9B9B} & &  & \\ \hline
\end{tabular}%
}
\caption{Planificación temporal inicial del proyecto.}
\label{tab:PlanificacionTemporal}
\end{table}

La planificación del proyecto se puede visualizar en la Tabla~\ref{tab:PlanificacionTemporal}, 
donde se observa las distintas fases del ciclo de vida. El desarrollo del proyecto empieza 
tras finalizar las clases lectivas debido a que el alumno estaba compaginando el trabajo con
los estudios. No obstante, dicha planificación no sufrió retrasos ni modificaciones significativas.

Con respecto a los gastos y materiales necesarios para el desarrollo del proyecto, se necesitó una
suscripción a \emph{Google Colab Pro}, un portátil personal de gama media, \emph{Google Drive 100GB} y
otros gastos varios. Además, se asume un salario de 12.56\officialeuro/hora, como para un investigador \emph{senior} o 
responsable I+D de una empresa tecnológica en España.

Respecto al servidor GPU, con las especificaciones actuales de \emph{Google}, 
se estima un coste aproximado de 10.000\officialeuro. Se asume una amortización de 2 años, 
lo que implica un pago diario de 13.70\officialeuro. El desglose total de los costes 
se puede ver en la siguiente Tabla \ref{tab:TotalGastos}.

\begin{table}[H]
\centering
\scriptsize
\begin{tabular}{ll}
\hline
\multicolumn{1}{|l|}{\cellcolor[HTML]{FFCB2F}{\textbf{Fecha inicio}}} & \multicolumn{1}{l|}{01/06/2025} \\ \hline
\multicolumn{1}{|l|}{\cellcolor[HTML]{FFCB2F}{\textbf{Fecha fin}}} & \multicolumn{1}{l|}{14/09/2025} \\ \hline
\multicolumn{1}{|l|}{\cellcolor[HTML]{FFCB2F}{\textbf{Duración}}} & \multicolumn{1}{l|}{136 días, 97 laborables} \\ \hline
\textbf{} & 
\end{tabular}
\caption{Total de horas y días trabajados.}
\label{tab:TotalTrabajado}
\end{table}

\begin{table}[H] 
  \centering
  \scriptsize
  \begin{tabular}{ll}
\hline
\rowcolor[HTML]{FFCB2F} 
\multicolumn{1}{|c|}{\cellcolor[HTML]{FFCB2F}{\textbf{Item}}} & \multicolumn{1}{c|}{\cellcolor[HTML]{FFCB2F}{\textbf{Costo}}} \\ \hline
\multicolumn{1}{|l|}{Salario} & \multicolumn{1}{l|}{10.048,00\officialeuro} \\ \hline
\multicolumn{1}{|l|}{Portátil de Gama Media} & \multicolumn{1}{l|}{700,00\officialeuro} \\ \hline
\multicolumn{1}{|l|}{Google Colab Pro} & \multicolumn{1}{l|}{55,95\officialeuro} \\ \hline
\multicolumn{1}{|l|}{Servidor GPU} & \multicolumn{1}{l|}{2.109,8\officialeuro} \\ \hline
\multicolumn{1}{|l|}{Google Drive 100GB} & \multicolumn{1}{l|}{10,00\officialeuro} \\ \hline
\multicolumn{1}{|l|}{Otros} & \multicolumn{1}{l|}{300,00\officialeuro} \\ \hline
\multicolumn{1}{|r|}{\cellcolor[HTML]{FFCB2F}{\textbf{Total}}} & \multicolumn{1}{l|}{ 13.223,75 \officialeuro} \\ \hline
\textbf{} & 
\end{tabular}
\caption{Estimación final de coste del proyecto.}
\label{tab:TotalGastos}
\end{table}
