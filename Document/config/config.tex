%\documentclass[a4paper,twoside,11pt,titlepage]{book}
\usepackage{listings}
\usepackage[utf8]{inputenc}
\usepackage[spanish, es-tabla]{babel}
\usepackage{eurosym}
%\usepackage{cite}

\usepackage[loadshadowlibrary]{todonotes}
\usepackage{float, subcaption}

\usepackage{colortbl}
\usepackage{multirow}
\usepackage{multicol}
\definecolor{M1}{RGB}{255,199,2}

\usepackage{amsmath}
\usepackage{dirtytalk}
\usepackage{xcolor}
\usepackage{tikz}
\usepackage{makecell}

\newcommand{\code}{\lstinline}

\usepackage{csquotes}
\usepackage{microtype}
\usepackage[backend=biber, defernumbers=true, citestyle=numeric-comp, bibstyle=ieee, sorting=none]{biblatex}

\DeclareBibliographyCategory{cited}
\AtEveryCitekey{\addtocategory{cited}{\thefield{entrykey}}}
% Configurando BibLaTeX
\DefineBibliographyStrings{spanish}{
  url = {URL},
  andothers={et ~al\adddot}
}

\addbibresource{bibliografia/bibliografia.bib}

\setlength{\parindent}{0pt}

\definecolor{ugr_yellow}{rgb}{1.0, 0.78, 0}

\newcommand{\addref}[2][1=]{\todo[shadow, color=black!40, caption={Pendiente Referencia: #1}]{#2}}
\newcommand{\towrite}[2][1=]{\todo[inline, shadow, color=gray!40, caption={A escribir: #1}]{En proceso: #2}}
\newcommand{\corrected}[2][1=]{\todo[inline, shadow, color=blue!40, caption={Propuesta: #1}]{#2}}
\newcommand{\changed}[2][1=]{\todo[shadow, color=yellow!40, caption={Cambios: #1}]{#2}}

\newcommand{\Enrique}[2][1=]{\todo[inline, shadow, color=red!40, caption={Enrique: #1}]{#2}}
\newcommand{\EnriqueL}[2][1=]{\todo[shadow, color=red!40, caption={Enrique: #1}]{#2}}
\newcommand{\Mesejo}[2][1=]{\todo[inline, shadow, color=orange!40, caption={Pablo: #1}]{#2}}
\newcommand{\MesejoL}[2][1=]{\todo[shadow, color=orange!40, caption={Pablo: #1}]{#2}}
\newcommand{\Brian}[2][1=]{\todo[inline, shadow, color=yellow!40, caption={Brian: #1}]{#2}}

%\usepackage[style=list, number=none]{glossary} %
%\usepackage{titlesec}
%\usepackage{pailatino}

\decimalpoint
\usepackage{dcolumn}
\newcolumntype{.}{D{.}{\esperiod}{-1}}
\makeatletter
\addto\shorthandsspanish{\let\esperiod\es@period@code}
\makeatother


%\usepackage[chapter]{algorithm}
\RequirePackage{verbatim}
%\RequirePackage[Glenn]{fncychap}
\usepackage{fancyhdr}
\usepackage{graphicx}
\usepackage{afterpage}

\usepackage{longtable}

\usepackage[pdfborder={000}, hidelinks]{hyperref} %referencia

% ********************************************************************
% Re-usable information
% ********************************************************************
\newcommand{\myTitle}{Metrología Monocular en Entornos No Controlados}
\newcommand{\mySubTitle}{}
\newcommand{\myTitleENG}{Single View Metrology in the Wild}
\newcommand{\mySubTitleENG}{}
\newcommand{\myDegree}{Máster en Ciencia de Datos e Ingeniería de Computadores}
\newcommand{\myName}{Brian Sena Simons}
\newcommand{\myProf}{Pablo Mesejo Santiago}
\newcommand{\myOtherProf}{Enrique Bermejo Nievas}
%\newcommand{\mySupervisor}{Put name here}
\newcommand{\myFaculty}{Escuela Técnica Superior de Ingenierías Informática y de
Telecomunicación}
\newcommand{\myFacultyShort}{E.T.S. de Ingenierías Informática y de
Telecomunicación}
\newcommand{\myDepartment}{Departamento de Ciencias de la Computación e Inteligencia Artificial }
\newcommand{\myUni}{\protect{Universidad de Granada}}
\newcommand{\myLocation}{Granada}
\newcommand{\myTime}{\today}
\newcommand{\myVersion}{Version 0.1}


\hypersetup{
pdfauthor = {\myName (briansenas@correo.ugr.es)},
pdftitle = {\myTitle},
pdfsubject = {},
% NOTE: update this once we finish
pdfkeywords = {palabra_clave1, palabra_clave2, palabra_clave3, ...},
pdfcreator = {LaTeX con el paquete pdflatex},
pdfproducer = {pdflatex}
}

%\hyphenation{}


%\usepackage{doxygen/doxygen}
%\usepackage{pdfpages}
\usepackage{url}
\usepackage{colortbl,longtable}
\usepackage[stable]{footmisc}
%\usepackage{index}

%\makeindex
%\usepackage[style=long, cols=2,border=plain,toc=true,number=none]{glossary}
%\makeglossary

% Definición de comandos que me son tiles:
%\renewcommand{\indexname}{Índice alfabético}
%\renewcommand{\glossaryname}{Glosario}

\pagestyle{fancy}
\fancyhf{}
\fancyhead[LO]{\leftmark}
\fancyhead[RE]{\rightmark}
\fancyhead[RO,LE]{\textbf{\thepage}}
\renewcommand{\chaptermark}[1]{\markboth{\textbf{#1}}{}}
\renewcommand{\sectionmark}[1]{\markright{\textbf{\thesection. #1}}}

\setlength{\headheight}{1.5\headheight}

\newcommand{\HRule}{\rule{\linewidth}{0.5mm}}
%Definimos los tipos teorema, ejemplo y definición podremos usar estos tipos
%simplemente poniendo \begin{teorema} \end{teorema} ...
\newtheorem{teorema}{Teorema}[chapter]
\newtheorem{ejemplo}{Ejemplo}[chapter]
\newtheorem{definicion}{Definición}[chapter]

\definecolor{gray97}{gray}{.97}
\definecolor{gray75}{gray}{.75}
\definecolor{gray45}{gray}{.45}
\definecolor{gray30}{gray}{.94}

\lstset{ frame=Ltb,
     framerule=0.5pt,
     aboveskip=0.5cm,
     framextopmargin=3pt,
     framexbottommargin=3pt,
     framexleftmargin=0.1cm,
     framesep=0pt,
     rulesep=.4pt,
     backgroundcolor=\color{gray97},
     rulesepcolor=\color{black},
     %
     stringstyle=\ttfamily,
     showstringspaces = false,
     basicstyle=\normalsize\ttfamily,
     commentstyle=\color{gray45},
     keywordstyle=\bfseries,
     %
     numbers=left,
     numbersep=6pt,
     numberstyle=\tiny,
     numberfirstline = false,
     breaklines=true,
   }
 
% minimizar fragmentado de listados
\lstnewenvironment{listing}[1][]
   {\lstset{#1}\pagebreak[0]}{\pagebreak[0]}

\lstdefinestyle{CodigoC}
   {
	basicstyle=\scriptsize,
	frame=single,
	language=C,
	numbers=left
   }
\lstdefinestyle{CodigoC++}
   {
	basicstyle=\small,
	frame=single,
	backgroundcolor=\color{gray30},
	language=C++,
	numbers=left
   }

 
\lstdefinestyle{Consola}
   {basicstyle=\scriptsize\bf\ttfamily,
    backgroundcolor=\color{gray30},
    frame=single,
    numbers=none
   }


\newcommand{\bigrule}{\titlerule[0.5mm]}


%Para conseguir que en las páginas en blanco no ponga cabecerass
\makeatletter
\def\clearpage{%
  \ifvmode
    \ifnum \@dbltopnum =\m@ne
      \ifdim \pagetotal <\topskip
        \hbox{}
      \fi
    \fi
  \fi
  \newpage
  \thispagestyle{empty}
  \write\m@ne{}
  \vbox{}
  \penalty -\@Mi
}
\makeatother

\usepackage{pdfpages}
